% ====================================================
% DEFINITIONS
% ====================================================
% definition of shape for abstract general nets (rectangle by default)

\makeatletter
\newcommand*{\rectx@anchor@bottom}[2]{%
  \anchor{p#1}{%
    \pgf@process{\northeast}%
    \pgf@xa=\pgf@x
    \pgf@process{\southwest}%
    \pgf@x=\dimexpr\pgf@x + (\pgf@xa-\pgf@x)*#1/#2\relax
  }%
}
\newcommand*{\declareshaperectx}[1]{%
  \pgfdeclareshape{rect#1}{%
    \inheritsavedanchors[from=rectangle]
    \inheritanchorborder[from=rectangle]
    \inheritanchor[from=rectangle]{north}
    \inheritanchor[from=rectangle]{north west}
    \inheritanchor[from=rectangle]{center}
    \inheritanchor[from=rectangle]{west}
    \inheritanchor[from=rectangle]{east}
    \inheritanchor[from=rectangle]{mid}
    \inheritanchor[from=rectangle]{mid west}
    \inheritanchor[from=rectangle]{mid east}
    \inheritanchor[from=rectangle]{base}
    \inheritanchor[from=rectangle]{base west}
    \inheritanchor[from=rectangle]{base east}
    \inheritanchor[from=rectangle]{south}
    \inheritanchor[from=rectangle]{south east}
    \inheritbackgroundpath[from=rectangle]
    \count@=\m@ne
    \@whilenum\count@<#1 \do{%
      \advance\count@\@ne
      \expandafter\rectx@anchor@bottom\expandafter{\the\count@}{#1}%
    }%
  }%
}
\makeatother

% ====================================================
% DECLARATION
% ====================================================

\declareshaperectx{8}
\declareshaperectx{16}
\declareshaperectx{32}
\declareshaperectx{64}
